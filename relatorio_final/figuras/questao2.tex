A principal diferencial do Maildir em relação ao Mailbox é a sua compatibilidade com sistemas de arquivos de rede, como, por exemplo, o NFS e o GFS. A compatibilidade pode ser explicada pelo fato desse formato utilizar um arquivo para cada mensagem de e-mail, enquanto que o Mbox armazena todas as mensagens em um único arquivo. Essa característica do Maildir evita a possibilidade de ocorrer um mecanismo conhecido como file locking, em que o arquivo é travado para gravação enquanto algum outro processo estiver gravando nele.

Essa diferença afeta o método de acesso ao sistema de arquivos. No Mbox, o acesso a um único arquivo exige desempenho para escrita e leitura seqüencial. No Maildir, os acessos são aleatórios entre os vários arquivos criados para cada mensagem. Assim, um sistema de arquivos que tem um bom desempenho para ler mensagens no formato Mbox, pode sofrer ao tentar ler essas mesmas mensagens no formato Maildir. Além de ter uma menor chance de corrupção e caso ocorra apenas uma mensagem é perdida e não todas.
